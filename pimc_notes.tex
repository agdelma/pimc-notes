\documentclass[prb,aps,amssym,nofootinbib,floatfix,notitlepage]{revtex4-1} 
\usepackage{graphicx,natbib}
\usepackage{amsmath}
\usepackage{amsfonts}
\usepackage{siunitx}
\usepackage{bbm}

\renewcommand{\vec}[1]{\boldsymbol{#1}}
\newcommand{\e}[1]{\mathrm{e}^{#1}}
\newcommand{\sgn}[1]{\mathrm{sgn}(#1)}
\renewcommand{\eqref}[1]{Eq.~(\ref{#1})}
\newcommand{\R}{\vec{R}}

\begin{document}
\title{Path Integral Monte Carlo and the Worm Algorithm in the Spatial
Continuum}
\author{Adrian Del Maestro}
\email{Adrian.DelMaestro@uvm.edu}
\affiliation{Department of Physics, University of Vermont, Burlington, VT 05405, USA}

\date{\today}
\maketitle

% ---------------------------------------------------------------------------------
\section{Expectation Values and the Partition Function}
% ---------------------------------------------------------------------------------

The goal of these lecture notes will be to describe a stochastically exact
quantum Monte Carlo method for computing the expectation value of observables
for systems of interacting particles in the spatial continuum described by a
Hamiltonian $\hat{\mathcal{H}}$ which conserves particle number.  We begin with
the general definition for the expectation value of some operator
$\hat{\mathcal{O}}$ in terms of a trace over configurations
%
\begin{equation}
    \langle \mathcal{O} \rangle = \frac{1}{\mathcal{Z}} \mathrm{Tr}\, 
    \hat{\mathcal{O}}\e{-\beta \hat{\mathcal{H}}} 
\label{eq:operatorExpectationValue}
\end{equation}
%
where $\beta = 1/k_{\mathrm{B}}T$ with $k_{\mathrm{B}}$ the Boltzmann constant
and the partition function $Z$ is given by
%
\begin{equation}
    \mathcal{Z} = \mathrm{Tr}\,\e{-\beta \hat{\mathcal{H}}} 
    = \mathrm{Tr}\,\hat{\rho} 
\label{eq:partitionFunction} 
\end{equation}
%
where 
%
\begin{equation}
\hat{\rho} \equiv \e{-\beta \hat{\mathcal{H}}} 
\end{equation}
%
is the density matrix.  In order to compute the trace in
\eqref{eq:partitionFunction} for a given system described by Hamiltonian
$\mathcal{H}$ we need to identify a set of convenient basis states
$|\alpha\rangle$ that can be efficiently sampled allowing us to write the
partition function as the direct sum 
%
\begin{equation}
    \mathcal{Z} =  \sum_{\alpha}\langle \alpha | \e{-\beta \hat{\mathcal{H}}}
    |\alpha \rangle.
\end{equation}
%
In the next section we will introduce the specific form of $|\alpha\rangle$
in terms of the spatial positions of interacting particles.

% ---------------------------------------------------------------------------------
\section{Path Integral Monte Carlo}
% ---------------------------------------------------------------------------------
The path-integral Monte Carlo method was first introduced by Ceperley, and a
comprehensive review can be found in Ref.~[\onlinecite{Ceperley:1995gr}]. Here
we will attempt to provide an introduction to the method with sufficient
details to allow for the creation of a simple code.

We are interested in a system of interacting particles described by the
general many-body Hamiltonian:
%
\begin{align}
    \hat{\mathcal{H}} &= \hat{\mathcal{T}} + \hat{\mathcal{V}} \nonumber \\
                      &= -\sum_i^N \frac{\hbar^2}{2m_i} \hat{\vec{\nabla}}_i^2 
    + \sum_{i=1}^N \hat{V}_{i} + \sum_{i < j} \hat{U}_{ij}.
\label{eq:Hamiltonian}
\end{align}
%
It will be convenient to work in first quantized notation, where the $N$
particles in the $d$-dimensional spatial continuum are located at positions
$\vec{r}_i$ with $i=1\ldots N$.  The first term in \eqref{eq:Hamiltonian}
corresponds to the kinetic energy $\hat{\mathcal{T}}$ where $m_i$ is the mass
of the $i^{\text{th}}$ particle. The external potential energy $V(\vec{r_i})$
only depends on the position of a single particle, while the two-body
interaction potential $U(\vec{r}_i-\vec{r}_j)$ is in general a function of the
vector displacement between them.  We will most often work with spherically
symmetric interaction potentials such that $U(\vec{r}_i - \vec{r}_j) =
U(|\vec{r}_i-\vec{r}_j|)$. A physical system of interest could include trapped
ultra-cold atoms, where $V(\vec{r}_i) \sim |\vec{r}_i|^2$ is a harmonic
trapping potential and the particles interact via an induced dipole-dipole
interaction $U(\vec{r}_i - \vec{r}_j) \sim |\vec{r}_i-\vec{r}_j|^{-3}$.

The most natural basis states $|\alpha\rangle$ in this case are just a
collection of the spatial locations of the $N$ particles, where in the case of
identical particles, the labels are fictitious. We will employ the convenient
short-hand notation 
%
\begin{equation}
    |\R\rangle \equiv |\vec{r}_1, \cdots, \vec{r}_N \rangle
\end{equation}
%
where particle conservation enforces the normalization constraint
%
\begin{equation}
\int \mathcal{D}\R\, |\R\rangle \langle \R | = \mathbbm{1}
\end{equation}
%
with 
%
\begin{equation}
    \int\mathcal{D} \R  \equiv \prod_{i=1}^N \int d \vec{r}_i.
\end{equation}
%
In the first-quantized spatial position basis, the partition function can be
written as a $N \times d$ dimensional integral
%
\begin{align}
    \mathcal{Z} &= \mathrm{Tr}\, \e{-\beta \hat{\mathcal{H}}}  \nonumber \\
                &= \int d\vec{r}_1 \cdots \int d\vec{r}_N \langle \vec{r}_1,
    \cdots, \vec{r}_N | \e{-\beta \hat{\mathcal{H}}}| \vec{r}_1, \cdots,
    \vec{r}_N \rangle \nonumber \\
&\equiv \int \mathcal{D} \R \langle \R | \e{-\beta \hat{\mathcal{H}}} | \R
    \rangle.
\label{eq:ZDR}
\end{align}
%
As terms like this will appear quite frequently, it will be useful to define
the elements of the density matrix at inverse temperature $\beta$ in the
spatial basis
%
\begin{equation}
    \rho(\R, \R'; \beta) \equiv \langle \R | \e{-\beta \hat{\mathcal{H}}} |
    \R'\rangle
\end{equation}
%
where all matrix elements are real and positive. We note that in the spatial
continuum, the Hilbert space is uncountably infinite, as the particles can take
on any position in $\mathbbm{R}^d$. This is an important observation that will
guide the strategy we choose in the duration of these notes.

As we have defined the potential operator $\hat{\mathcal{V}}$ to be diagonal in
the position basis, it would be extremely convenient if we could decompose the
density matrix into a product of terms containing $\hat{\mathcal{T}}$ and
$\hat{\mathcal{V}}$ However, we know that
$[\hat{\mathcal{T}},\hat{\mathcal{V}}] \ne 0$ and thus
%
\begin{align}
    \hat{\rho} &= \e{-\beta(\hat{\mathcal{T}} + \hat{\mathcal{V}})} \nonumber
    \\
&\ne \e{-\beta\hat{\mathcal{T}}}\e{-\beta\hat{\mathcal{V}}}.
\end{align}
%
In fact, by employing the Baker-Campbell-Hausdorff formula we know:
%
\begin{equation}
    \e{\hat{A}+\hat{B}} = \e{\hat{A}}\e{\hat{B}}\e{-\frac{1}{2}[\hat{A},\hat{B}]}
\end{equation}
%
and thus 
%
\begin{equation}
    \hat{\rho} = \e{-\beta\hat{\mathcal{T}}}\e{-\beta\hat{\mathcal{V}}} +
    \mathrm{O}\left(\beta^2\right).
\end{equation}
%
with the error diverging in the interesting low temperature limit $\beta \to
\infty$.  


\bibliographystyle{apsrev4-1}
\bibliography{refs}

\end{document}
